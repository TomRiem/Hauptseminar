\chapter{Conclusion}

This work has set itself the goal of describing the Riemannian Trust-Region Symmetric Rank-One method and illustrating its performance in the programming language Julia. For that, the Euclidean counterpart was introduced and the most important statements about convergence were presented. For generalizing the Euclidean Trust-Region Symmetric Rank-One method into the Riemannian setup, basic principles about Riemannian optimization were introduced. The main part of the work summarized how the SR1 update for matrices can be generalized to a self-adjoint rank-one update for operators on a tangent space within a Riemannian Trust-Region approach, which led to the Riemannian Trust-Region Symmetric Rank-One method. The main convergence statements of this method were presented, by which it was seen that the local $n + 1$-step q-superlinear rate of convergence is preserved. The SR1 update for operators on the tangent space for use in the already existing Riemannian Trust-Region method was implemented in the package \lstinline!Manopt.jl!. An experiment was conducted in the programming language Julia in which was found that the Riemannian Trust-Region Symmetric Rank-One method converges for the chosen optimization problem, but not as fast or efficient as the Riemannian Trust-Region Newton method. Nevertheless, this newly implemented method offers an interesting variant for solving Riemannian optimization problems, whose application may be a reasonable choice for specific problems in the future if the number of allocations can be reduced. 