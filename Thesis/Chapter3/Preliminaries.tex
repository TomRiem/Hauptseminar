\section{Riemannian Trust-Region Methods}

% Gradient, Hessian, retraction, pullback, models, trust-region subproblem, quasi-Newton operator, general trust-region method, global convergence

From now on we consider Riemannian optimization problems, which consider finding an optimum of a real-valued function $f$ defined on a Riemannian manifold $\mathcal{M}$, i.e.
\begin{equation*}
    \min f(x), \quad x \in \mathcal{M}.
\end{equation*}
From now on we assume that $\mathcal{M}$ is a $n$-dimensional geodesically complete Riemannian manifold. We further assume that the manifold $\mathcal{M}$ is embedded in a real-valued space (e.g. $\mathcal{M} \subseteq \mathbb{R}^m$) and connected. Further we assume that $f \colon \; \mathcal{M} \to \mathbb{R}$ is a twice continuously differentiable function, i.e. $f \in C^2(\mathcal{M})$. \\
Along a curve $\gamma(\alpha) = \retract{x_k}(\alpha \eta_k)$ on the manifold $\mathcal{M}$, which is defined by a chosen retraction $\retractionSymbol \colon \; \tangent{} \to \mathcal{M}$ (see \cite[Definition~4.1.1]{AbsilMahonySepulchre:2008}). Therefore the choice of a computationally efficient retraction is an important decision in the design of high-performance numerical algorithms on manifolds \cite[p.~54]{AbsilMahonySepulchre:2008}.

Here $\mathcal{H}_k [\cdot]$ tries to approximate the action of the Riemannian Hessian $\operatorname{Hess} f(x_k) [\cdot]$ (see \cite[Definition~5.5.1]{AbsilMahonySepulchre:2008}) and $\mathcal{B}_k [\cdot]$ approximates the action of ${\operatorname{Hess} f(x_k)}^{-1} [\cdot]$. At the end of each iteration, these operators are updated to new operators $\mathcal{H}_{k+1}, \mathcal{B}_{k+1} \colon \; \tangent{x_{k+1}} \to \tangent{x_{k+1}}$ on the upcoming tangent space $\tangent{x_{k+1}}$. It is required that these satisfy some form of the Riemannian quasi-Newton equation. 

By introducing the tangent vector $s_k = \parallelTransportDir{x_k}{\alpha_k \eta_k}(\alpha_k \eta_k) \in \tangent{x_{k+1}}$, which represents the connection between the two iterates $x_k$ and $x_{k+1}$, and the tangent vector $y_k = \operatorname{grad} f(x_{k+1}) - \parallelTransportDir{x_k}{\alpha_k \eta_k}(\operatorname{grad} f(x_k)) \in \tangent{x_{k+1}}$, which represents the difference of the gradients in $\tangent{x_{k+1}}$, the Riemannian quasi-Newton equation is given by 
\begin{equation*}
    \mathcal{H}_{k+1} [s_k] = y_k 
\end{equation*}

To be able to create rank-one operators, we introduce the musical isomorphism $\flat \colon \; \tangent{x_{k+1}} \ni \eta_{x_{k+1}} \mapsto \eta^{\flat}_{x_{k+1}} \in \cotangent{x_{k+1}}$ (see \cite[p.~6]{BergmannHerzogLouzeiroSilvaTenbrinckVidalNunez:2020:1}). Put simply, it means: $\eta^{\flat}_{x_{k+1}} \in \cotangent{x_{k+1}}$ represents the flat of $\eta_{x_{k+1}} \in \tangent{x_{k+1}}$, i.e., $\eta^{\flat}_{x_{k+1}} \colon \; \tangent{x_{k+1}} \to \mathbb{R}, \;  \xi_{x_{k+1}} \mapsto \eta^{\flat}_{x_{k+1}}[\xi_{x_{k+1}}] = g_{x_{k+1}} (\eta_{x_{k+1}}, \xi_{x_{k+1}})$. This generalizes the notion of the transpose. As noted earlier, there are several quasi-Newton update formulae for operators that have been generalized from the Euclidean case to the Riemannian setup. 

\begin{equation}\label{Riemanniantrsubproblem}
    s_k = \arg \min_{\lVert s \rVert_{x_k} \leq \Delta_k} m_k( s ) = \arg \min_{\lVert s \rVert_{x_k} \leq \Delta_k} f(x_k) + g_{x_k}(\operatorname{grad} f(x_k), s ) + \frac{1}{2} g_{x_k}( s , \mathcal{H}_k [ s ] ).
\end{equation}

