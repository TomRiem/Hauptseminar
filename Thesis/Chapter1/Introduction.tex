\chapter{Introduction}

Optimization on Riemannian manifolds, also called Riemannian optimization, concerns finding an optimum of a real-valued function $f$ defined over a manifold. It can be thought of as unconstrained optimization on a constrained space. As such, optimization algorithms on manifolds are not fundamentally different from classical algorithms for unconstrained optimization in $\mathbb{R}^n$. On an Euclidean space, various methods of solving unconstrained optimization problems are known. \\
Trust-Region methods are among others the most important numerical optimization methods in solving nonlinear unconstrained optimization problems, which form to lineserach methods an alternative class of algorithms that combine desirable global convergence properties with a local superlinear rate of convergence. Trust-Region methods construct a quadratic model $m_k$ of the objective function $f$ around the current iterate $x_k$ and produce a candidate for the next iterate by minimizing the model $m_k$ within a region where it is “trusted”. Depending on the agreement between the objective function $f$ and the quadratic model $m_k$ at the candidate, the size the of trust-region is updated and the candidate is accepted or rejected. \\
Especially the quadratic term of the model is important for a fast rate of convergence. For this, a compromise is sought between computational costs in practice and good convergence properties. Among several strategies, the Symmetric Rank-One, short SR1, update known from the quasi-Newton methods is favored in view of its simplicity and because it preserves symmetry without unnecessarily enforcing positive definiteness. With the SR1 update, an approximation of the Hessian matrix is generated in the course of the iterations by adding up rank-one matrices, which receive information about the curvature of the objective function in each iteration. The resulting Trust-Region Symmetric Rank-One method has been shown to be very effective in optimization calculations. Therefore, a generalization of this method for application to optimization problems on Riemannian manifolds is desirable, which requires that many definitions are reconsidered. This reconsideration is crucial because the ideas are not extended simply from the Euclidean setup. \\
The main purpose of this work is to show how the Trust-Region Symmetric Rank-One method is generalized for the application to Riemannian optimization problems and to show the performance of a Riemannian Trust-Region Symmetric Rank-One method implemented in the package \lstinline!Manopt.jl! (available at \url{https://manoptjl.org}, \cite{Bergmann:2019}) by comparing the results with those obtained with a Riemannian Trust-Region Newton method, also implemented in \lstinline!Manopt.jl!, which uses the Hessian operator as a quadratic term. 

