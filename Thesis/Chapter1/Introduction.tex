\chapter{Introduction}

Optimization on Riemannian manifolds, also called Riemannian optimization, concerns finding an optimum of a real-valued function $f$ defined over a manifold. It can be thought of as unconstrained optimization on a constrained space. As such, optimization algorithms on manifolds are not fundamentally different from classical algorithms for unconstrained optimization in $\mathbb{R}^n$. On an Euclidean space, various methods of solving unconstrained optimization problems are known. \\
Trust-region methods are among others the most important numerical optimization methods in solving nonlinear unconstrained optimization problems, which form to lineserach methods an alternative class of algorithms that combine desirable global convergence properties with a local superlinear rate of convergence. Trust-region methods construct a quadratic model $m_k$ of the objective function $f$ around the current iterate $x_K$ and produce a candidate new iterate by minimizing the model $m_k$ within a region where it is “trusted”. Depending on the agreement between the objective function $f$ and $m_k$ at the candidate, the size the of trust-region is updated and the candidate is accepted as upcomming iterate or rejected. Especially the quadratic term of the model is important for a fast rate of convergence. For this, a compromise is sought between computational costs in practice and good convergence properties. The Symmetric Rank-One, short SR1, update known from the quasi-Newton methods has been shown to be very effective in optimization calculations, especially when used in a trust-region approach. The concepts of these algorithms can be used for the Riemannian optimization if many definitions are reconsidered. This reconsideration is crucial because the ideas are not extended simply from the Euclidean setup. \\
The main purpose of this work is to show how the Trust-Region Symmetric Rank One is generalized and the performance of a Riemannian Trust-Region SR1 method implemented in the package \lstinline!Manopt.jl! (available at \url{https://manoptjl.org}, \cite{Bergmann:2019}) is compared with that of a Riemannian BFGS method.
